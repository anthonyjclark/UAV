\documentclass{article}\usepackage[]{graphicx}\usepackage[]{color}
% maxwidth is the original width if it is less than linewidth
% otherwise use linewidth (to make sure the graphics do not exceed the margin)
\makeatletter
\def\maxwidth{ %
  \ifdim\Gin@nat@width>\linewidth
    \linewidth
  \else
    \Gin@nat@width
  \fi
}
\makeatother

\definecolor{fgcolor}{rgb}{0.345, 0.345, 0.345}
\newcommand{\hlnum}[1]{\textcolor[rgb]{0.686,0.059,0.569}{#1}}%
\newcommand{\hlstr}[1]{\textcolor[rgb]{0.192,0.494,0.8}{#1}}%
\newcommand{\hlcom}[1]{\textcolor[rgb]{0.678,0.584,0.686}{\textit{#1}}}%
\newcommand{\hlopt}[1]{\textcolor[rgb]{0,0,0}{#1}}%
\newcommand{\hlstd}[1]{\textcolor[rgb]{0.345,0.345,0.345}{#1}}%
\newcommand{\hlkwa}[1]{\textcolor[rgb]{0.161,0.373,0.58}{\textbf{#1}}}%
\newcommand{\hlkwb}[1]{\textcolor[rgb]{0.69,0.353,0.396}{#1}}%
\newcommand{\hlkwc}[1]{\textcolor[rgb]{0.333,0.667,0.333}{#1}}%
\newcommand{\hlkwd}[1]{\textcolor[rgb]{0.737,0.353,0.396}{\textbf{#1}}}%
\let\hlipl\hlkwb

\usepackage{framed}
\makeatletter
\newenvironment{kframe}{%
 \def\at@end@of@kframe{}%
 \ifinner\ifhmode%
  \def\at@end@of@kframe{\end{minipage}}%
  \begin{minipage}{\columnwidth}%
 \fi\fi%
 \def\FrameCommand##1{\hskip\@totalleftmargin \hskip-\fboxsep
 \colorbox{shadecolor}{##1}\hskip-\fboxsep
     % There is no \\@totalrightmargin, so:
     \hskip-\linewidth \hskip-\@totalleftmargin \hskip\columnwidth}%
 \MakeFramed {\advance\hsize-\width
   \@totalleftmargin\z@ \linewidth\hsize
   \@setminipage}}%
 {\par\unskip\endMakeFramed%
 \at@end@of@kframe}
\makeatother

\definecolor{shadecolor}{rgb}{.97, .97, .97}
\definecolor{messagecolor}{rgb}{0, 0, 0}
\definecolor{warningcolor}{rgb}{1, 0, 1}
\definecolor{errorcolor}{rgb}{1, 0, 0}
\newenvironment{knitrout}{}{} % an empty environment to be redefined in TeX

\usepackage{alltt}
\IfFileExists{upquote.sty}{\usepackage{upquote}}{}
\begin{document}

\section{Introduction}

\subsection{What is Rasberry Pi?}

The Raspberry Pi is an tiny computer, that includes a tiny processor, a bit of memory, a slot for an SD card, and some input/output jacks, e.g. HDMI, USB, headphone, camera, and pin header for various sensors.

\subsection{Why use Raspberry Pi?}

The Pi has a lot of functionality and flexibility for develop monitoring of environmental parameters. 

\subsection{Uses in Environmental Science}

\subsubsection{Weather and Climate Change}

\subsubsection{Air Pollution Monitoring}

\subsubsection{Soil Water Monitoring and Irrigation Control}

%https://fyi.extension.wisc.edu/cropirrigation/files/2015/03/Methods.to_.Monitor.Soil_.Moisture.pdf

\subsubsection{Conservation} 

Conservation biologists use a wide range of instruments to track and monitor wildlife (camera traps, active (radio) and passive (RIF) transmitters). In addition the use of cameras are used to evaluate plant health and diversity (spectral analysis). 



\subsection{Resources}

\subsection{Raspberry Pi}

The Raspberry was created to help non-technical youth to learn computer and robotics. The Raspberry Pi was an unexpected success and now the Pi is one of the most important minaturized computers for a wide range of projects.  

The %[RaspberryPi.org](https://www.raspberrypi.org/) has tutorials, software updates, and example projects.

\subsubsection{Unpacking the Raspberry Pi}

When you recieve your Pi, plan on spending about 1-2 hours setting it up which include the following steps:

\begin{enumerate}

\item Unpack Kit Contents
\item Put Pi in case and add heat sinks (video instructions for the case at https://www.canakit.com/pi-case)
\item Connect to keyboard, mouse and monitor. Make sure the HDMI plug is in the correct mini-HDMI socket and the monitor is configured to get a signal from the port being used. 
\item Insert pre-loaded SD card
\item Plug-in Pi, you'll see a rainbow screen for a minute and then a installation menu. 
\item Install the Rasbian operating system only. This will take 10 minutes. Read the little windows so you know some of the resources associated with the operating system. The installations stalls at the end, where it says 100\%. Be patient, it will finish on it's own and reboot. 
\item Once the Rasbian OS starts, you'll see four raspberries at the top and then you'd get some prompts to set up the OS. 
\item I suggest you keep the password as default for now, select the langauge, keyboard type, and time zone. 
\item Next you'll need to get connected to the internet. Select your modem and enter password to connect.
\item Then you'll get a prompt to check for updates. Yes, there will be updates. This will take another 10 minutes. About 1/2 way the screen goes completely blank. I am not sure why, but it's rather disconcerting. \item Just like with a desktop, you will need to shut down the Raspberry Pi before removing it from power.  You can either do this through the raspberry icon in the top left corner of your console under the "shutdown" option, or you can type "$sudo shutdown -h now" in the console.  Once you have done this you can remove your pi from power/disconnect.

\end{enumerate}

\subsubsection{Video Tutorials}

%[Amazon Prime -- Introducing Rasberry Pi History, Models, and Uses](https://www.amazon.com/gp/video/detail/B07ZTR6ST2/ref=atv_dp_share_cu_r)

\subsubsection{Troubleshooting}

\subsubsubsection{Screensaver}

To keep the screen from going blank after ten minutes of inaction, use "$sudo apt-get install xscreensaver".  If this doesn't work, make sure your sudo is up to date with "$sudo apt-get update".  More information/different methods are available at: %[Pi Forum:  Disable Screensaver is Raspbian](https://www.raspberrypi.org/forums/viewtopic.php?f=91&t=57552)

\subsection{Remote Access}

** NOTE: I ADDED LOTS OF INFORMATION BELOW (AND A FEW POINTS ABOVE) BUT I'VE NEVER WORKED IN LATEX BEFORE.  I TRIED TO MIMIC YOUR SYNTAX AS MUCH AS POSSIBLE BUT IT WON'T LET ME KNIT, SO I CLEARLY DID SOMETHING INCORRECTLY.  MAYBE WE CAN TROUBLESHOOT DURING OUR MEETING TOMORROW **

It doesn't take too long (~10 minutes once you get the hang of it) to be able to work remotely on your microprocessor via a tablet or laptop.  More information at %[Remote Access with Raspberry Pi Tutorial](https://pythonprogramming.net/remote-access-raspberry-pi-tutorials/) 

\begin{enumerate}

\item Begin by making sure all of your packages are up-to-date (even if you updated while setting up your OS, you will still need to do this).  Do this with the code "$sudo apt-get update" and then "$sudo apt-get upgrade".  This will take several minutes.
\item Enable the SSH server by typing "$if config" into the console.  This will lead you to a screen with several options; select option five, "interfacing options," and then select option two for "SSH."  Enable the server, and then exit this screen.
\item Install the packages necessary to remotely access your screen: "$sudo apt-get remove xrdp vnc4server tightvncserver", "$sudo apt-get install tightvncserver", and finally "$sudo apt-get install xrdp".
\item Type "$ifconfig" into your console to find your Raspberry Pi's inet address under the "wlan" subcategory.  This should be a string of numbers that looks something like "192.168.XXX.X.XXX."  You will need this to access your Raspberry Pi remotely.
\item Connecting to Mac or Linux OS: open the terminal and add "ssh pi@[YOUR PI's INET ADDRESS]".  [Note: the author does not have a Mac or Linux OS, so this has not been tested].
\item Connecting to Windows OS: this is a bit more involved; you will need to download an SSH client.  The Microsoft "Remote Desktop Connection" app is available for free, and is pretty intuitive to use.  Download this, and follow the instructions to set it up.  Once it is installed, all you will need to do is enter the inet address you found above into the text box, and it will open a page for you to add your username and password (which at this point should still be "pi" and "raspberry").  Then your console will open.
\subitem If it won't connect: make sure your microprocessor and the computer you are attempting to accesss it from are on the same wifi; make sure your microprocessor is connected and hasn't gone to sleep; make sure you've properly enabled all of the packages (I had to try to run xrdp twice to get it to work the first time).  
\item Still working on: remotely accessing w/o HDMI connection, mirroring between microprocessor console and remote access console (more information to come).

\subsection{Projects General-purpose input-output Examples}

\subsubsection{Weather and Climate}

%\href{Thermocouples Sensors}{https://www.omega.com/en-us/resources/thermocouples]}

\subsubsection{Air Quality}

\subsubsection{Sensors}

Particulate Matter

NOx

O3

CO

Temperature and Humidity

\subsubsection{Software}

- Python Script

\subsubsection{Soil Moisture}

[cheap version](https://tutorials-raspberrypi.com/measuring-soil-moisture-with-raspberry-pi/)

[Raspberry 3 and Capacitance Sensors](https://www.switchdoc.com/2018/11/tutorial-capacitive-moisture-sensor-grove/)

[moisture-sensor-dfrobot](https://tutorials-raspberrypi.com/raspberry-pi-capacitive-spoil-moisture-sensor-dfrobot-gravity/)


\subsubsection{Wildlife}

Poacher Cam v7, Chris Kline, panthera.org

\end{document}
